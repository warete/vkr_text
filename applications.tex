\newpage
\addcontentsline{toc}{section}{Приложение А (обязательное) Программный код} 
\begin{center} 
    \textbf{Приложение А}
    \\
    \textbf{(обязательное)}
    \\
    \textbf{Программный код}
\end{center} 
\vspace{8mm}
\par
Листинг А.1 -- Код программы для классификации данных компьютерного моделирования яркостной температуры
\vspace{8mm}
\large
\begin{verbatim}
import numpy as np
import pandas as pd
from sklearn import svm
import matplotlib.pyplot as plt
from sklearn import neighbors
from sklearn.naive_bayes import GaussianNB
from sklearn.utils import shuffle
modelFolder = '075'
def calcParams(testing, predict):
    zdorovieTesting = 0
    bolnieTesting = 0
    for item in testing:    
        if item == 0:
            zdorovieTesting += 1
        else:
            bolnieTesting += 1
    zdoroviePredict = 0
    bolniePredict = 0
    for item in predict:    
        if item == 0:
            zdoroviePredict += 1
        else:
            bolniePredict += 1	
	chuvstv = bolniePredict / bolnieTesting
	specifich = zdoroviePredict / zdorovieTesting
	return chuvstv,specifich
	
data = pd.read_csv(modelFolder + '/data.csv', 
delimiter=',', 
names=['0ртм', '1ртм', '2ртм', '3ртм', '4ртм', '5ртм', '6ртм', '7ртм', '8ртм', 
'0ик', '1ик', '2ик', '3ик', '4ик', '5ик', '6ик', '7ик', '8ик', 'target'])
trainingTemperatures = data[['0ртм', '1ртм', '2ртм', '3ртм', '4ртм', '5ртм', 
'6ртм', '7ртм', '8ртм', '0ик', '1ик', '2ик', '3ик', '4ик', '5ик', '6ик',
'7ик', '8ик']]
trainingTemperatures = shuffle(trainingTemperatures)
trainingTemperatures.reset_index(inplace=True, drop=True)
data.head()
testingData = pd.read_csv(modelFolder + '/testing.csv', 
delimiter=',', 
names=['0ртм', '1ртм', '2ртм', '3ртм', '4ртм', '5ртм', '6ртм', '7ртм', '8ртм', 
'0ик', '1ик', '2ик', '3ик', '4ик', '5ик', '6ик', '7ик', '8ик', 'target'])
testingTemperatures = testingData[['0ртм', '1ртм', '2ртм', '3ртм', '4ртм',
'5ртм', '6ртм', '7ртм', '8ртм', '0ик', '1ик', '2ик', '3ик', '4ик', '5ик',
'6ик', '7ик', '8ик']]

#SVM
clf = svm.SVC(gamma='scale')
clf.fit(trainingTemperatures, data.target)

classes = clf.predict(testingTemperatures)

testingData.target
res = testingData.target == classes
trueCount = 0
for el in res:
	if el == True:
		trueCount += 1

print('Правильно определено:', trueCount, '\nНеправильно:', len(res) - trueCount)
chuvstv, specifich = calcParams(testingData.target, classes)
print('Чувствительность:', chuvstv, '\nСпецифичность:', specifich);

labels = 'Совпадает', 'Не совпадает'
values = [trueCount, len(res) - trueCount]
explode = (0.1, 0)

fig1, ax1 = plt.subplots()
ax1.pie(values, labels=labels, autopct='%1.1f%%',
shadow=True, startangle=90)
ax1.axis('equal')

plt.show()

#KNN
clf = neighbors.KNeighborsClassifier(20, weights='uniform')
clf.fit(trainingTemperatures, data.target)

classes = clf.predict(testingTemperatures)

testingData.target
res = testingData.target == classes
trueCount = 0
for el in res:
	if el == True:
		trueCount += 1

print('Правильно определено:', trueCount, '\nНеправильно:', len(res) - trueCount)
chuvstv, specifich = calcParams(testingData.target, classes)
print('Чувствительность:', chuvstv, '\nСпецифичность:', specifich);

labels = 'Совпадает', 'Не совпадает'
values = [trueCount, len(res) - trueCount]
explode = (0.1, 0)

fig1, ax1 = plt.subplots()
ax1.pie(values, labels=labels, autopct='%1.1f%%',
shadow=True, startangle=90)
ax1.axis('equal')

plt.show()

#NB
gnb = GaussianNB()
clf = gnb.fit(trainingTemperatures, data.target)

classes = clf.predict(testingTemperatures)

testingData.target
res = testingData.target == classes
trueCount = 0
for el in res:
	if el == True:
		trueCount += 1

print('Правильно определено:', trueCount, '\nНеправильно:', len(res) - trueCount)
chuvstv, specifich = calcParams(testingData.target, classes)
print('Чувствительность:', chuvstv, '\nСпецифичность:', specifich);

labels = 'Совпадает', 'Не совпадает'
values = [trueCount, len(res) - trueCount]
explode = (0.1, 0)

fig1, ax1 = plt.subplots()
ax1.pie(values, labels=labels, autopct='%1.1f%%',
shadow=True, startangle=90)
ax1.axis('equal')

plt.show()
\end{verbatim}
\vspace{8mm}
\Large
\par
Листинг А.2 -- Код программы для отображения результатов частотного анализа
\vspace{8mm}
\large
\begin{verbatim}
import numpy as np
import pandas as pd
from sklearn import svm
import matplotlib.pyplot as plt
import matplotlib
from sklearn import neighbors
from sklearn.naive_bayes import GaussianNB
from collections import Counter
from collections import OrderedDict
from scipy.interpolate import spline

modelFolder = 'all'
data = pd.read_csv(modelFolder + '/data.csv', 
delimiter=',', 
names=['0ртм', '1ртм', '2ртм', '3ртм', '4ртм', '5ртм', '6ртм', '7ртм', '8ртм', 
'0ик', '1ик', '2ик', '3ик', '4ик', '5ик', '6ик', '7ик', '8ик', 'target'])

point = '0ртм'
filter1 = data.target == 0
fr1 = OrderedDict(sorted(dict(Counter(data.loc[filter1][point])).items()))
names1 = list(fr1.keys())
values1 = list(fr1.values())

x_smooth1 = np.linspace(min(names1), max(names1), 50)
y_smooth1 = spline(names1, values1, x_smooth1)

filter2 = data.target == 1
fr2 = OrderedDict(sorted(dict(Counter(data.loc[filter2][point])).items()))
names2 = list(fr2.keys())
values2 = list(fr2.values())

x_smooth2 = np.linspace(min(names2), max(names2), 50)
y_smooth2 = spline(names2, values2, x_smooth2)


fig, axs = plt.subplots(2, 1, figsize=(10, 10), sharey=True)
plt.subplots_adjust(wspace=0, hspace=0.17)
axs[0].bar(names1, values1)
#axs[0].plot(x_smooth1, y_smooth1, color='tab:orange')
axs[0].tick_params(axis='both', which='major', labelsize=15)
axs[0].set_title('Здоровые пациенты')

axs[1].bar(names2, values2)
#axs[1].plot(x_smooth2, y_smooth2, color='tab:orange')
axs[1].tick_params(axis='both', which='major', labelsize=15)
axs[1].set_title('Больные пациенты')
plt.savefig(point + '.png')
plt.show()
\end{verbatim}
\vspace{8mm}
\Large
\newpage
\addcontentsline{toc}{section}{Приложение Б (обязательное) Лист задания}
фффф
\newpage
фффф
\newpage
\addcontentsline{toc}{section}{Приложение В (обязательное) Полученные при выполнении работы компетенции}
ффф