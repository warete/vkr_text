\newpage
\section{\Large Полученные при выполнении работы компетенции}

\par
В ходе выполнения выпускной квалификационной работы были освоены 
следующие компетенции:

\begin{enumerate} 
  \item Компетенция ОК-1 (способность использовать основы философских знаний для формирования мировоззренческой позиции) была освоена при изучении материалов по областям применения алгоритмов машинного обучения в первой главе.
  
  \item Компетенция ОК-2 (способность анализировать основные этапы и закономерности исторического развития общества для формирования гражданской позиции) была освоена при изучении и обзоре литературы в первой главе.
  
  \item Компетенция ОК-3 (способность использовать основы экономических знаний в различных сферах деятельности) была освоена при планировании работ по разработке приложения.
  
  \item Компетенция ОК-4 (способность использовать основы правовых знаний в различных сферах деятельности) была освоена при проектировании приложения.
  
  \item Компетенция ОК-5 (способность к коммуникации в устной и письменной формах на русском и иностранном языках для решения задач межличностного и межкультурного взаимодействия) была освоена при получении консультаций у научного руководителя и научного консультанта.
  
  \item Компетенция ОК-6 (способность работать в коллективе, толерантно воспринимая социальные, этнические, конфессиональные и культурные различия) была освоена при изучении теоретического материала с научным руководителем и научным консультантом.
  
  \item Компетенция ОК-7 (способность к самоорганизации и самообразованию) была освоена при составлении плана будущих работ в первой главе, выполнении практических заданий по 
проектированию и программированию, а также при изучении теоретического материала.
  
  \item Компетенция ОК-8 (способностью использовать методы и средства физической культуры для обеспечения полноценной социальной и профессиональной деятельности) была освоена при разработке приложения по составленному плану.
  
  \item Компетенция ОК-9 (способность использовать приемы оказания первой помощи, методы защиты в условиях чрезвычайных ситуаций) была освоена при изучении техники безопасности при работе в лаборатории.
  
  \item Компетенция ОПК-1 (способность инсталлировать программное и аппаратное обеспечение для информационных и автоматизированных систем) была освоена при подготовке к проектированию и разработке приложения.
  
  \item Компетенция ОПК-2 (способность осваивать методики использования программных средств для решения практических задач) была освоена при реализации приложения во второй главе.
  
  \item Компетенция ОПК-3 (способность разрабатывать бизнес-планы и технические задания на оснащение отделов, лабораторий, офисов компьютерным и сетевым оборудованием) была освоена при составлении плана работ.
  
  \item Компетенция ОПК-4 (способность участвовать в настройке и наладке программно-аппаратных комплексов) была освоена при отладке и тестировании приложения.
  
  \item Компетенция ОПК-5 (способность решать стандартные задачи профессиональной деятельности на основе информационной и библиографической культуры с применением информационно-коммуникационных технологий и с учетом основных требований) 
была освоена при изучении теоретического материала в первой.
  
  \item Компетенция ПК-1 (способность разрабатывать модели компонентов информационных систем, включая модели баз данных и модели интерфейсов «человек - электронно-вычислительная машина») была освоена при проектировании и реализации 
приложения во второй главе.
  
  \item Компетенция ПК-3 (способность обосновывать принимаемые проектные решения, осуществлять постановку и выполнять эксперименты по проверке их корректности и эффективности) была освоена при проектировании интерфейса и архитектуры приложения во второй главе.
  
  \item  Компетенция ПК-5 (способность сопрягать аппаратные и программные средства в составе информационных и автоматизированных систем) была освоена при разработке приложения во второй главе.
  
  \item  Компетенция ПК-6 (способность подключать и настраивать модули ЭВМ и периферийного оборудования) была освоена при работе над выпускной квалификационной работой.
  
\end{enumerate}