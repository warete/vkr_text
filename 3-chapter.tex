\newpage
\section{\Large Применение методов машинного обучения для классификации данных компьютерного моделирования}
Работа получившейся программы была протестирована на тестовом наборе данных из общей выборки(рисунок~\ref{ris:data_example}). Рассмотрим детальнее каждый из этапов по работе с программой.

\subsection{Обучение модели и классификация тестовой выборки}
Сначала нужно загрузить CSV-файл с результатами компьютерного моделирования биотканей. Для этого нужно нажать на кнопку «Выбрать» возле специального поля и выбрать файл на компьютере(рисунок~\ref{ris:upload_data_screenshot}).
\imgh{1\linewidth}{upload_data_screenshot}{Скриншот интерфейса программы при выборе файла с данными для загрузки}
\par
После выбора файла и нажатия на кнопку «Загрузить» файл будет загружен на сервер. Если файл загрузился успешно, то пользователь увидит сообщение как на рисунке~\ref{ris:upload_success}. После загрузки произойдет переинициализация всех используемых в приложении методов библиотеки Scikit-learn и будут очищены файлы с сохраненными моделями.
\imgh{0.65\linewidth}{upload_success}{Сообщение об успешной загрузке файла на сервер}
\par
Также в результате обновления файла были обновлены данные статистических метрик и отрисованы графики (рисунок~\ref{ris:statistics_metrics}). Исходя из этих данных можно сделать первые выводе об используемых при обучении данных. 
\imgh{1\linewidth}{statistics_metrics}{Скриншот графиков со статистическими метриками, вычисленных на основе загруженных данных}
\par
На примере графика с частотным распределением опухолей по точкам можно заметить, что больше всего опухолей расположено в крайних точках с номерами 4 и 8, а меньше всего моделей с опухолями в соседних точках с номерами 6 и 7.
\par
Следующим этапом после загрузки файла следует выбор метода классификации из списка, обучение модели и запуск классификации для тестовой выборки. Для всех перечисленных действий есть отдельные элементы управления (рисунок~\ref{ris:fir_predict_interface}). 
\imgh{1\linewidth}{fir_predict_interface}{Элементы управления для выбора размера тестовой выборки, обучения модели и классификации тестовой выборки}

\subsection{Определение класса «Болен»/«Здоров» и точки с опухолью}
Результатом обучения и классификации является круговая диаграмма с точностью определения класса «Болен»/«Здоров». На рисунке~\ref{ris:bayes_accuracy} изображен пример круговой диаграммы с точностью для наивного байесовского классификатора.
\imgh{1\linewidth}{bayes_accuracy}{Круговая диаграмма с точностью классификации для наивного байесовского классификатора}
После тестирования нескольких методов классификации были получены следующие результаты по точности определения класса:
\begin{itemize}
	\item[-] SVM -- 66.7\%;
	\item[-] k-ближайших соседей -- 65.4\%;
	\item[-] Наивный байесовский классификатор -- 70\%;
	\item[-] Ансамблевый метод bagging и SVM -- 68.8\%;
	\item[-] Стохастический градиентный спуск -- 56.3\%.
\end{itemize}
\par
Исходя из представленных результатов, можно сделать вывод, что для данной выборки лучше всего отработал наивный байесовский классификатор и ансамблевый метод bagging в сочетании с SVM.
\par
Так же есть возможность определить класс (диагноз) по данным пациента. После заполнение всех нужных полей и нажатия на кнопку «Диагностировать» буду получены данные о классе и точки, к которой ближе всего расположена опухоль. На рисунке~\ref{ris:location_accuracy} приведен пример резульата дианостирования на температурных данных одной из моделей, не попавшей в обучающую выборку. После определения точки с опухолью строится круговая диаграмма с точностью. 
\imgh{1\linewidth}{location_accuracy}{Круговая диаграмма с точностью классификации при определении локализации опухоли}
Как видно на диаграмме -- получилось добиться точности определения класса равной 67.5\%. Возможно этот результат получится улучшить с помощью большего объема обучающей выборки или использования другого алгоритма классификации.

