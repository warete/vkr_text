\newpage
\section{Заключение}
В данной работе были рассмотрены сферы деятельности и основные задачи, где используются методы машинного обучения, а также некоторые из популярных библиотек языка программирования Python для решения таких задач. Был описан принцип работы таких алгоритмов классификации как метод опорных векторов (SVM), k-ближайших соседей и наивный байесовский классификатор.
\par
Было реализовано клиент-серверное приложение для для классификации данных компьютерного моделирования яркостной температуры. При разработке использовались такие библиотеки языка Python как Flask и Scikit-learn. При разработке клиентской части использовался фреймворк VueJs. Были разработы Unit-тесты для упрощения доработок программы в будущем. Температурные данные были разбиты на обучающую и тестовую выборки и классифицированы с помощью получившейся программы.
\par
Исходя из результатов классификации моделей был сделан вывод, что точность классификации данных сильно зависит от используемого алгоритма. Лучше всего в проведенных экспериментах себя показал метод опорных векторов (SVM) и k-ближайших соседей.