\newpage
\section{\Large Методы и области применения машинного обучения}\vspace{-7mm}

\subsection{Обзор литературы}
\par
Статья Полякова М.В., Хоперскова А.В. «Математическое моделирование пространственного распределения радиационного поля в биоткани: определение яркостной температуры для диагностики», опубликованная в Вестнике Волгоградского государственного университета, посвящена проведению имитационных экспериментов по моделированию динамики температурных и радиационных полей в биотканях молочной железы. В работе вместо традиционно используемых моделей с однородными параметрами используются вычислительные модели максимально приближенные к реалистичной геометрической структуре тканей с неоднородными характеристиками~\cite{polyakovKhoperskov}.
\par
В статье Веснина С.Г., Седанкина М.К. «Миниатюрные антенны-\\аппликаторы для микроволновых радиотермометров медицинского назначения», опубликованная в журнале «Биомедицинская радиоэлектроника», описывается анализ миниатюрных антенн-аппликаторов, предназначенных для измерения собственного излучения тканей человека с помощью микроволновых радиотермометров. Приведены простые аппроксимационные формулы для распределения температуры в молочной железе при наличиии злокачественной опухоли~\cite{vesninMinAntenn}.
\par
В работе Van Ongeval Ch. «Digital mammography for screening and diagnosis of breast cancer: an overview» обсуждается цифровая телемаммография как новая техника для диагностирования заболеваний молочных желез. Также в данной работе детально рассматривается проблема практической реализации различных систем для визуализации телемаммографической диагностики, высокой стоимости обследования и высокой квалификации специалистов-радиологов~\cite{vanOngeval}.
\par
Работа Nisreen I. Yassin, Shaimaa Omran, Enas M. F. El Houby, Enas M. F. El Houby, Hemat Allam «Machine Learning Techniques for Breast Cancer Computer Aided Diagnosis Using Different Image Modalities: A Systematic Review» посвящена опыту медиков в диагностики и обнаружении рака молочной железы с использованием алгоритмов машинного обучения на основе визуализированных данных обследования пациентов. Целью работы является исследование современного уровня техники в отношении систем компьютерной диагностики и обнаружения рака молочной железы~\cite{nisrenml}.
\par
В статье Левшинского В., Полякова М., Лосева А., Хоперскова А. \\ «Verification and Validation of Computer Models for Diagnosing Breast Cancer Based on Machine Learning for Medical Data Analysis» рассмотрен поход проверки результатов моделирования физических процессов в биотканях с использованием глубокого анализа и машинного обучения. При обучении моделей используются данные измерений температуры пациентов согласно методу радиотермометрии. Так же в работе выделяются на основе набора данных для обучения новые признаки, похожие на те, которые используют медики при обследовании пациентов~\cite{lev-polyakov-losev-hoperskov}.
\par
В работе Рамсундара Б., Истмана П., Уолтерса П., Панде В. «Глубокое обучение в биологии и медицине» обсуждается применение глубокого обучения в популярных направлениях современных исследований, а особенно в биологии и медицине. Работа содержит описание архитектуры алгоритмов в машинном обучении для применения в задачах данных сферах, а так же некоторые практические примеры по использованию~\cite{ramsundar-deep-learning}.
\par
Статья Jian Ma, Pengchao Shang, Chen Lu, Safa Meraghni «A portable breast cancer detection system based on smartphone with infrared camera» посвящена разработке системы обнаружения рака молочной железы с использованием смартфона с инфракрасной камерой. Для обследования использовался метод инфракрасной термографии и алгоритм классификации k-ближайших соседей. Авторам удалось достигнуть точности определения наличия заболевания больше 98\%~\cite{mobile-breast-cancer-detection}.
\par
В части работ рассмотрен метод микроволновой радиотермометрии и его применение при обследовании рака молочных желез. Так же в некоторых работах рассмотрены способы применения машинного обучения для диагностирования различных заболеваний, в том числе онкологических. Рассмотрим далее подробно области применения и основные алгоритмы в машинном обучении.

\subsection{Области применения}
Использование алгоритмов машинного обучения позволяет решать задачи в различных сферах деятельности человека, таких как недвижимость, сельское хозяйство, экономика, а так же медицина. По данным агенства Frost \& Sullivan спрос на разработки, в которых используется машинное обучение в медицине, увеличивается с каждым годом примерно на 40\%~\cite{habrbigdatamedicine}. Такие разработки могут использоваться как для диагностики заболеваний, так  и для биохимических исследований.
\par
Методы машинного обучения активно применяются при медицинском сканировании различных типов, таких как УЗИ или компьютерная томография. Благодаря алгоритмам распознавания образов на изображениях есть возможность анализировать результаты таких исследований и указывать на проблемные участки. Также возможно определение диагноза пациента по различным его параметрам и результатам исследования. Но программное обеспечение, использующее данные алгоритмы пока не может заменить полностью работу медиков и используется в основном при первичных исследованиях в качестве экспертных систем.
\par
При компьютерном моделировании алгоритмы машинного обучения могут использоваться для валидации получившихся данных, или прогнозирования течения каких-либо физических процессов.

\subsection{Методы классификации в машинном обучении}
Классификация данных состоит из прогнозирования определенного результата на основе уже известных данных. Чтобы предсказать результат, алгоритм обрабатывает данные, содержащие набор атрибутов и соответствующий каждому набору результат, обычно называемый атрибутом прогнозирования цели или классом~\cite{hetal2016}. Формируется модель алгоритма, которая пытается обнаружить отношения между атрибутами, которые позволили бы предсказать результат~\cite{kumbhar}. Такая процедура называется обучением модели, а набор данных, используемый для этого -- тестовой выборкой~\cite{Mirmozaffari}.
\par
Следующим шагом после обучения модели является прогнозирование -- процедура определения класса, при которой используется набор данных с неизвестными классами. Такой набор данных, который содержит тот же набор атрибутов, за исключением атрибута прогнозирования, часто называют тестовой выборкой~\cite{tprogeralgorithms}.
\par
Алгоритм анализирует входные данные и выдает прогноз. Точность прогноза определяет, насколько «хорош» алгоритм. Например, в медицинской базе данных обучающий набор должен иметь соответствующую информацию о пациенте, записанную ранее, где атрибутом прогноза является наличие или отсутствие у пациента проблем со здоровьем.
\par
Для определения того, какой именно алгоритм использовать для конкретной задачи можно воспользоваться схемой, изображенной на рисунке~\ref{ris:scikitlearn-map}. Исходя из того, что в текущей задаче используется не тестовая информация и имеется 160 примеров, то были выбраны алгоритмы SVM, k-ближайших соседей и наивный байесовский классификатор, описанные ниже.
\\
\imgh{1\linewidth}{scikitlearn-map}{Схема для определения алгоритма классификации для конкретной задачи}
\subsection{Метод опорных векторов}
Метод опорных вектором или SVM -- это метод статистической классификации~\cite{kristianini}. Он широко используется для задач различного рода и хорошо себя в них показывает~\cite{crammer}.
\par
Основной идеей метода является представление атрибутов данных в виде векторов и переход в пространство более высокой размерности, чем получившееся на этапе представления векторами. Затем ищется гиперплоскость с максимальным зазором в пространстве между объектами разных классов~\cite{rashka}~\cite{statnikov}.
\par
На рисунке~\ref{ris:svm_example} показан пример классификации методом SVM. Красной линией выделена как раз та самая гиперплоскость, четко разделяющая объекты разных классов друг от друга.
\\
\imgh{0.6\linewidth}{svm_example}{Пример классификации методом SVM}
\par
Алгоритм может использоваться с одним из следующих видов ядер~\cite{crammer}:
\begin{itemize}
	\item[-] Полиномиальное (однородное) $k(x,x^{'}) = (x \cdot x^{'})^{d}$;
	\item[-] Полиномиальное (неоднородное) $k(x,x^{'}) = (x \cdot x^{'} + 1)^{d}$;
	\item[-] Радиальная базисная функция $k(x,x^{'}) = exp(-\gamma ||x - x^{'}||^{2})$, для $\gamma > 0$;
	\item[-] Радиальная базисная функция Гаусса $k(x,x^{'}) = exp\left ( -\frac{||x - x^{'}||^{2}}{2\sigma^{2}} \right )$;
	\item[-] Сигмоид $k(x,x^{'}) = tanh(kx\cdot x^{'} + c)$.
\end{itemize}
\subsection{Метод k-ближайших соседей}
Алгоритм k-ближайших соседей является простым статистическим алгоритмом обучения, в котором объект классифицируется своими соседями. При классификации таким методом объект относится к классу, наиболее распространенному среди его k-ближайших соседей~\cite{potapov}~\cite{flah}. Пример классификации приведен на рисунке~\ref{ris:knn_example}, где в качестве классифицируемого объекта используется прямоугольник и существует несколько объектов известных классов -- белые точки и черные. Замерив расстояние от объекта до его соседей с различными классами и основываясь на методе k-ближайших соседей данный объект будет отнесен к классу черный точек, а не белых.
\\
\imgh{0.6\linewidth}{knn_example}{Пример классификации методом k-ближайших соседей}
\par
При нахождении атрибутов учитывается значимость атрибутов и часто применяется прием растяжения осей, демонстрируемый в формуле~\eqref{eq:eq1}. Использование данного приема снижает ошибку классификации.
\begin{equation}\label{eq:eq1}
D_{E} = \sqrt{3(x_{A} - y_{A})^{2} + (x_{B} - y_{B})^{2}},
\end{equation}
где $x_{A}, y_{A}$ -- значения атрибута A в наборе данных, $x_{B}, y_{B}$ -- значения атрибута B.
\par
Данный алгоритм возможно применять как для данных с маленьким количеством атрибутов, так и с достаточно большим. Важным моментом при работе с алгоритом является определение функции расстояния между значениями. Примером такой функции может быть евклидово расстояние -- формула~\eqref{eq:eq2}.
\begin{equation}\label{eq:eq2}
D_{E} = \sqrt{\sum_{i}^{n}(x_{i} - y_{i})^{2}},
\end{equation}
где $x_{i}, y_{i}$ -- значения атрибутов в наборе данных.
\subsection{Наивный байесовский классификатор}
Наивный байесовский классификатор является простым вероятностным классификатором и основывается на применении теоремы Байеса со строгими предположениями о независимости~\cite{potapov}~\cite{danilovsv}. Хотя наивный байесовский классификатор редко применим к большинству реальных задач, но зачастую в определенных задачах он демонстрирует хорошие результаты и часто конкурирует с более сложными методами, такими как SVM и классификационным деревьями~\cite{Mirmozaffari}. Классификация данным методом очень зависит от распределения зависимостей атрибутов, а не от самих зависимостей~\cite{juravlev}.
\par
Вероятностная модель классификатора:
\begin{equation}\label{eq:eq3}
p(C|F_{1},...,F_{n}) = \frac{p(C)p(F_{1},...,F_{n}|C)}{p(F_{1},...,F_{n})},
\end{equation}
где $C$ -- класс модели, а $F_{i}$ -- классифицируемые модели~\cite{Mirmozaffari}.
\par
Использование формулы~\eqref{eq:eq3} при классификации дает минимально значение среднего риска или математичского ожидания ошибки:
\begin{equation}\label{eq:eq4}
R(a) = \sum_{y\in Y}\sum_{\varsigma\in Y}\lambda_{y}P_{y}P_{x,y}\left\lbrace a(x)=\varsigma|y \right\rbrace ,
\end{equation}
где $\lambda_{y}$ -- цена ошибки при отнесении объекта класса $Y$ к какому-либо другому классу.


\subsection{Популярные библиотеки с реализацией методов машинного обучения}
На текущий момент существует множество готовых реализаций алгоритмов машинного обучения и не имеет смысла делать то же самое с нуля, если задача не имеет каких-то особенностей, делающих невозможным использование готовых библиотек. Каждая из библиотек, рассматриваемых в работе, хороша в своей области, успешно используется в решении задач и проверена временем. Рассмотрим некоторые из популярных библиотек для языка программирования Python по данным рейтинга на GitHub (рисунок~\ref{ris:github_stars})
\\
\imgh{0.95\linewidth}{github_stars}{Популярные пакеты Python для машинного обучения по данным рейтинга на GitHub}
\subsubsection{TensorFlow}
Самой популярной и масштабной по применению является библиотека TensorFlow, используемая для глубокого машинного обучения~\cite{gudfellow}. Библиотека разрабатывается в тесном сотрудничестве с компанией Google и применяется в большинстве их проектов где используется машинное обучение. Библиотека использует систему многоуровневых узлов, которая позволяет вам быстро настраивать, обучать и развертывать искусственные нейронные сети с большими наборами данных.
\par
Библиотека хорошо подходит для широкого семейства техник машинного обучения, а не только для глубокого машинного обучения. Программы с использованием TensorFlow можно компилировать и запускать как на CPU, так и на GPU. Также данная библиотека имеет обширный встроенный функционал логирования, собственный интерактивный визуализатор данных и логов~\cite{muller}.
\subsubsection{Keras}
Keras используется для быстрого прототипирования систем с использованием нейронных сетей и машинного обучения. Пакет представляет из себя высокоуровневый API, который работает поверх TensorFlow или Theano. Поддерживает как вычисления на CPU, так и на GPU
\subsubsection{Scikit-learn}
Scikit-learn -- это одна из самых популярных библиотек для языка Python, в которой реализованы основные алгоритмы машинного обучения, такие как классификация различных типов, регрессия и кластеризация данных. Библиотека распространяется свободно и является бесплатной для использования в своих проектах~\cite{rashka}.
\par
Данная библиотека создана на основе двух других -- NumPy и SciPy, имеющих большое количество готовых реализаций часто используемых математических и статистических функций. Библиотека хорошо подходит для простых и средней сложности задач, а также для людей, которые только начинают свой путь в изучении машинного обучения.
\subsubsection{PyTorch}
PyTorch -- это популяный пакет Python для глубокого машинного обучения, который можно использовать для расширения функционала совместно с такими пакетами как NumPy, SciPy и Cython. Главной функцией PyTorch является возможность вычислений с использованием GPU. Отличается высокой скоростью работы и удобным API-интерфейсом расширения с помощью своей логики, написанной на C или C++.
\subsubsection{Theano}
Theano -- это библиотека, в которой содержится базовый набор инструментов для машинного обучения и конфигурирования нейросетей. Так же у данной библиотеки есть встроенные методы для эффективного вычисления математических выражений, содержащих многомерные массивы~\cite{rashka}.
\par
Theano тесно интегрирована с библиотекой NumPy, что дает возможность просто и быстро производить вычисления. Главным преимуществом библиотеки является возможность использования GPU без изменения кода программы, что дает преимущество при выполнении ресуркоемких задач. Также возможно использование динамической генерации кода на языке программирования C~\cite{douson}.